\documentclass[12pt, a4paper]{article}

% Packages for formatting
\usepackage[T1]{fontenc}
\usepackage[utf8]{inputenc}
\usepackage{lmodern}
\usepackage{graphicx}
\usepackage{hyperref}
\usepackage{geometry}
\usepackage{fancyhdr}
\usepackage{titlesec}
\usepackage{tocloft}
\usepackage{float}
\usepackage{caption}
\usepackage{booktabs}
\usepackage{array}
\usepackage{amsmath}
\usepackage{setspace}
\usepackage{parskip}

% Set up geometry
\geometry{
    a4paper,
    left=25mm,
    right=25mm,
    top=25mm,
    bottom=25mm,
}

% Set up header and footer
\pagestyle{fancy}
\fancyhf{}
\fancyhead[L]{CS3210 Final - Solar Simulator}
\fancyfoot[C]{\thepage}

% Section formatting
\titleformat{\section}{\normalfont\Large\bfseries}{\thesection}{1em}{}
\titleformat{\subsection}{\normalfont\large\bfseries}{\thesubsection}{1em}{}

% Hyperlink setup
\hypersetup{
    colorlinks=true,
    linkcolor=blue,
    filecolor=magenta,
    urlcolor=cyan,
}

% URL style
\urlstyle{same}

\begin{document}

% Title Page
\begin{titlepage}
    \centering
    \vspace*{4cm}
    {\Huge\bfseries CS3210 Final - Solar Simulator \par}
    \vspace{1cm}
    {\Large User Manual \par}
    \vfill
    {\large \today \par}
\end{titlepage}

\tableofcontents
\thispagestyle{empty}
\clearpage

% Start content
\setcounter{page}{1}

\section{Introduction}

Welcome to the \textbf{Solar System Simulator}, an interactive 3D visualization
of our solar system built with Three.js. This user manual will guide you
through installing, using, and understanding the features of the simulator.
Whether you're an astronomy enthusiast or just curious about our cosmic
neighborhood, this simulator is designed to provide an immersive educational
experience.

\section{Background Information}

Understanding the vastness and complexity of our solar system can be
challenging. This simulator aims to bridge that gap by visualizing the
positions, movements, and characteristics of the planets in an interactive 3D
environment.

\subsection{Key Concepts}

\begin{itemize}
    \item \textbf{Orbital Mechanics}: The simulator uses Keplerian orbital elements to calculate the positions of planets over time, providing an accurate representation of their movements.
    \item \textbf{Axial Tilt}: Each planet's rotation axis is tilted relative to its orbital plane, affecting the simulation of seasons and day/night cycles.
    \item \textbf{Astronomical Units (AU)}: Distances in space are immense, so the simulator scales down measurements using AU, where 1 AU is the average distance from the Earth to the Sun (~149.6 million kilometers).
\end{itemize}

\section{Installation}

\subsection{Prerequisites}

Before installing the Solar System Simulator, ensure that you have the
following software installed on your system:

\begin{itemize}
    \item \textbf{Node.js} (latest LTS version)
    \item \textbf{npm} (comes with Node.js)
    \item \textbf{Git LFS} (Git Large File Storage)
    \item \textbf{Modern Web Browser} (with WebGL 2.0 support, e.g., Chrome, Firefox)
\end{itemize}

\newpage

\subsection{Git LFS Setup}

This project uses Git LFS to manage large texture files. Follow these steps to
set up Git LFS:

\begin{enumerate}
    \item \textbf{Install Git LFS}:

          \begin{itemize}
              \item \textbf{Windows}:
                    \begin{verbatim}
winget install Git.LFS
        \end{verbatim}
              \item \textbf{macOS}:
                    \begin{verbatim}
brew install git-lfs
        \end{verbatim}
              \item \textbf{Linux}:
                    \begin{verbatim}
sudo apt install git-lfs
        \end{verbatim}
          \end{itemize}

    \item \textbf{Initialize Git LFS}:

          \begin{verbatim}
git lfs install
    \end{verbatim}

\end{enumerate}

\subsection{Cloning the Repository}

Open your terminal and execute the following commands:

\begin{enumerate}
    \item \textbf{Clone the repository}:

          \begin{verbatim}
git clone https://github.com/justinfeatherstone/
featherstoneplesciafarhat_3210_final.git
    \end{verbatim}

    \item \textbf{Navigate to the project directory}:

          \begin{verbatim}
cd featherstoneplesciafarhat_3210_final
    \end{verbatim}

    \item \textbf{Pull LFS files (if not already done)}:

          \begin{verbatim}
git lfs pull
    \end{verbatim}

\end{enumerate}

\subsection{Installing Dependencies}

Install the required npm packages:

\begin{verbatim}
npm install
\end{verbatim}

\newpage

\subsection{Running the Simulator}

Start the development server:

\begin{verbatim}
npx vite
\end{verbatim}

Open your web browser and navigate to \url{http://localhost:5173} to view the
simulator.

\section{Getting Started}

Upon launching the simulator, you will see the Sun at the center, surrounded by
the orbiting planets.

\subsection{User Interface Overview}

The simulator's interface consists of four main panels:

\begin{enumerate}
    \item \textbf{Navigation Panel}: Located on the left side, allows quick selection of planets and reset view.
    \item \textbf{Planet Information Panel}: On the right side, displays detailed information about the selected planet.
    \item \textbf{Time Control Panel}: At the bottom center, controls simulation speed and allows pausing/resuming time.
    \item \textbf{Orbit Controls Panel}: Next to the planet information panel, allows toggling the visibility of orbital paths.
\end{enumerate}

\begin{figure}[H]
    \centering
    \caption{Simulator Interface Overview}
\end{figure}

\section{Using the Simulator}

\subsection{Navigating the Solar System}

\begin{itemize}
    \item \textbf{Rotate View}: Click and drag to rotate around the current focus point.
    \item \textbf{Zoom}: Use the mouse wheel or trackpad scrolling to zoom in and out.
    \item \textbf{Pan}: Right-click and drag to pan the camera.
\end{itemize}

\subsection{Interacting with Planets}

\begin{itemize}
    \item \textbf{Select a Planet}: Click on a planet or select it from the navigation panel to focus on it.
    \item \textbf{Access Planet Information}: View detailed information in the planet information panel.
    \item \textbf{Navigate Between Planets}: Use the left (\texttt{<}) and right (\texttt{>}) arrows in the planet information panel.
\end{itemize}

\subsection{Adjusting Time Speed}

\begin{itemize}
    \item \textbf{Time Slider}: Adjust the simulation speed using the slider in the time control panel.
    \item \textbf{Pause/Resume}: Click the "Pause" button to halt time progression; click again to resume.
\end{itemize}

\subsection{Controlling Orbits}

\begin{itemize}
    \item \textbf{Toggle Orbit Visibility}: Use the checkboxes in the orbit controls panel to show or hide orbital paths.
    \item \textbf{Customize Orbits}: Select specific planets whose orbits you wish to view or hide.
\end{itemize}

\section{Technical Challenges and Design Decisions}

\subsection{Accurate Orbital Mechanics}

\textbf{Challenge}: Implementing realistic planet movements using Keplerian orbital elements.

\textbf{Solution}: Calculated each planet's position based on its orbital parameters, accounting for eccentricity, inclination, and other factors. Solved Kepler's Equation numerically to simulate true orbital paths.

\subsubsection{Design Decision}

Chose a balance between accuracy and performance by adjusting the number of
segments in orbit lines based on each planet's eccentricity.

\subsection{Performance Optimization}

\textbf{Challenge}: Rendering complex 3D scenes with detailed textures and effects without compromising performance.

\textbf{Solution}:

\begin{itemize}
    \item Implemented efficient algorithms for orbital calculations.
    \item Applied Levels of Detail (LOD) techniques to manage resource-intensive
          elements.
    \item Optimized textures and utilized Git LFS to handle large files.
\end{itemize}

\subsubsection{Design Decision}

Prioritized essential visual elements and introduced performance monitoring to
ensure smooth operation across devices.

\subsection{Visual Realism}

\textbf{Challenge}: Achieving visually appealing and realistic representations of celestial bodies.

\textbf{Solution}:

\begin{itemize}
    \item Used high-resolution textures for planet surfaces.
    \item Created custom shader materials for effects like Earth's day/night cycle and
          the Sun's glow.
    \item Added visual effects such as lens flares, atmospheric haze, and rotating clouds
          on Earth.
\end{itemize}

\subsubsection{Design Decision}

Implemented custom shaders for key planets to enhance realism, accepting the
added complexity for improved user experience.

\subsection{User Interface Design}

\textbf{Challenge}: Designing an intuitive and responsive UI that provides necessary information without overwhelming the user.

\textbf{Solution}:

\begin{itemize}
    \item Organized information into collapsible sections within the planet information
          panel.
    \item Implemented dynamic controls that update based on user interaction.
    \item Ensured responsiveness across different screen sizes using CSS Grid and
          flexible layouts.
\end{itemize}

\subsubsection{Design Decision}

Placed essential controls within easy reach and made advanced options
accessible but unobtrusive.

\subsection{Time Manipulation}

\textbf{Challenge}: Allowing users to control the simulation speed while maintaining accurate physics.

\textbf{Solution}:

\begin{itemize}
    \item Implemented a time scaling system that adjusts the simulation's time step based
          on user input.
    \item Ensured that orbital calculations are synchronized with the time scale for
          consistency.
\end{itemize}

\subsubsection{Design Decision}

Allowed negative time scales for reverse simulation and provided intuitive
feedback on time changes.

\section{Features Overview}

\subsection{Realistic Planet Textures}

High-quality visuals provide an immersive experience, using textures sourced
from reputable astronomical databases.

\subsection{Accurate Orbits}

Planets move according to real-world astronomical data, calculated using
Keplerian orbital elements.

\subsection{Interactive Controls}

Full control over navigation, time manipulation, and visual elements enhances
user engagement.

\subsection{Informative Panels}

Detailed information about each planet is presented in an organized and
accessible manner.

\subsection{Visual Effects}

Incorporation of lens flares, glowing sun effects, rotating clouds, and
atmospheric phenomena add to the realism of the simulation.

\subsection{Responsive Design}

The user interface is optimized for various screen sizes, ensuring a consistent
experience across devices.

\section{Troubleshooting}

\subsection{Planets Appear Without Textures}

\textbf{Cause}: Texture files are missing due to Git LFS not being properly configured.

\textbf{Solution}:

Ensure Git LFS is installed and initialized before cloning the repository.

\begin{enumerate}
    \item Install Git LFS (\textit{see Section 3.2}).
    \item Initialize Git LFS:

          \begin{verbatim}
git lfs install
    \end{verbatim}

    \item Clone the repository and pull LFS files:

          \begin{verbatim}
git clone https://github.com/yourusername/solar-system-simulator.git
git lfs pull
    \end{verbatim}

\end{enumerate}

\subsection{Performance Issues}

\textbf{Cause}: The simulation may be resource-intensive on some devices.

\textbf{Solution}:

\begin{itemize}
    \item Close unnecessary applications to free up system resources.
    \item Use a modern browser with good WebGL support (e.g., latest versions of Chrome
          or Firefox).
    \item Lower the simulation speed using the time control panel.
\end{itemize}

\subsection{Controls Not Responding}

\textbf{Cause}: The canvas may not have focus, or there could be an error in the application.

\textbf{Solution}:

\begin{itemize}
    \item Click on the canvas area to ensure it has focus.
    \item Open the browser console to check for any error messages.
    \item Ensure all dependencies are properly installed.
\end{itemize}

\subsection{Orbit Paths Not Visible}

\textbf{Cause}: Orbit visibility may be toggled off in the orbit controls panel.

\textbf{Solution}:

\begin{itemize}
    \item Open the orbit controls panel.
    \item Ensure the checkboxes next to the desired planets are checked.
\end{itemize}

\section{Credits and Resources}

\subsection{Textures}

\begin{itemize}
    \item \textbf{Solar System Scope Textures}: \href{https://www.solarsystemscope.com/textures/}{www.solarsystemscope.com/textures} (Attribution 4.0 International License)
    \item \textbf{JHT's Planetary Pixel Emporium}: \href{https://planetpixelemporium.com/}{planetpixelemporium.com} (Used with permission)
\end{itemize}

\subsection{Libraries and Tools}

\begin{itemize}
    \item \textbf{Three.js}: JavaScript 3D library used for rendering.
    \item \textbf{Vite}: Development server and build tool for modern web projects.
    \item \textbf{Git LFS}: Git extension for versioning large files.
    \item \textbf{Font Awesome}: Icon library for UI elements.
    \item \textbf{Google Fonts}: Typography resources for the user interface.
\end{itemize}

\subsection{Further Reading}

\begin{itemize}
    \item \textbf{Keplerian Orbital Elements}: \href{https://en.wikipedia.org/wiki/Orbital_elements}{Wikipedia Article}
    \item \textbf{Astronomical Units}: \href{https://solarsystem.nasa.gov/astrophysics/solar-system/}{NASA Solar System Exploration}
    \item \textbf{Three.js Documentation}: \href{https://threejs.org/docs/}{threejs.org/docs}
\end{itemize}

\section{Conclusion}

We hope this simulator provides you with an engaging way to explore our solar
system. The combination of accurate data and interactive features aims to both
educate and inspire curiosity about the cosmos.

For any questions or feedback, please reach out!

\vfill

\begin{center}
    \textit{Enjoy your journey through space!}
\end{center}

\end{document}
